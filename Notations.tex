% !TEX root = main.tex
The Transmitter energy arrival instants are marked by $t_i$'s with energy $\mathcal{E}_i$'s for $i \in \{0,1..\}$. The receiver spends a constant $P_{r}$ amount of power to be in '\textit{on}' state during which it can accept bits from the Transmitter. When it is in '\textit{off}' state it cannot accept bits and uses no power as well. Hence each energy arrival (say $E$) at the receiver can be viewed as adding $\mathcal{R}_i=\dfrac{E}{P_{r}}$ amount of time for which the receiver can be \textit{on}. The instances of energy arrival (or so called '\textit{time}' arrival) at the receiver is denoted by $r_i$. The maximum amount of time for which the receiver (and hence the Transmitter) can be \textit{on} assuming no energy is arriving at the receiver after time '$t$' is given by function $\mathcal{R}(t)$. It can be easily seen that $\mathcal{R}(t)=\underset{i:r_i\le t}{\sum}\mathcal{R}_i$. Similarly the maximum energy harvested at the transmitter till time '$t$' is given by function $\mathcal{E}(t)=\underset{i=0:t_i\le t}{\sum}\mathcal{E}_i$. The rate at which bits are transmitted with power '$p$' is given by function $g(p)$. This function is assumed to follow the following properties as proposed in \cite{Yang} 
\begin{align}
&P1) &&g(0)=0\text{ and }\lim_{x\rightarrow \infty} g(x)= \infty,\label{property_0_infty}
\\
&P2) &&g(x)\text{ is concave in nature with } x,\label{property_concave}
\\
&P3) &&g(x)\text{ is increasing with } x,\label{property_increasing}
\\ 
&P4) &&g(x)/(x) \text{ is monotonically decreasing with } x\nonumber
\\
&    &&\text{ and } \lim_{x\rightarrow \infty} g(x)/x= 0.\label{property_decreasing}
\end{align}

%Any policy used for transmission is represented in form of $\{\textbf{p},\textbf{s},N\}$, where $N\in \mathbb{N}$, $\textbf{p}=[p_1\ p_2\ ..\ p_N]$ and $\textbf{s}=[s_1\ s_2\ ..\ s_{N+1}]$. The transmitter starts transmitting at time instant $s_1$ with power $p_1$ and continues till time $s_2$. From $s_2$ it transmits with power $p_2$ and so on. The policy ends at time $s_{N+1}$. 

Suppose in a transmission policy, the transmitter starts transmitting at time $s_1$ with power $p_1$ and continues till $s_2$. From $s_2$ it transmits with power $p_2$ and so on. The last section of transmission begins at time $s_N$ with power $p_N$, where $N\in \mathbb{N}$. The transmission ends at time $s_{N+1}$. The receiver is kept \textit{on} during the time $[s_i,s_{i+1}]$ when $p_i\ne 0$, $\forall \ i=1,2..,N$, and kept \textit{off} otherwise. Such a policy is represented by the notation $\{\textbf{p},\textbf{s},N\}$, where $\textbf{p}=[p_1\ p_2\ ..\ p_N]$ and $\textbf{s}=[s_1\ s_2\ ..\ s_{N+1}]$. The energy used by this policy at the transmitter upto time '$t$' is given by the function $U(t)$ and the bits sent is represented by $B(t)$. Clearly,
\begin{align}
&U(t)=\sum_{i=1}^{j} p_i(s_{i+1}-s_i)+p_{j+1}(t-s_j) ,
\\
&B(t)=\sum_{i=1}^{j} g(p_i)(s_{i+1}-s_i)+g(p_{j+1})(t-s_j),
\\
&\text{where }j=\argmax_{i} \hspace{2mm}t_i<t.\nonumber
\end{align}
The function $\mathcal{P}(a,b)=\dfrac{\mathcal{E}(b^- )-U(a)}{b-a}$,  ($a>b$) denotes the maximum constant power with which transmitter can transmit from time $a$ to $b$, given that $U(a)$ amount of energy is already used upto time $a$. $a^-$ denotes the limiting value which approaches $a$ from left hand side.

%For convenience of presentation, we also follow the following convention : we use the notation $\stackrel{L1}{=}$ or $\stackrel{(1)}{=}$ or $\stackrel{P1}{=}$ or $\stackrel{T1}{=}$ to indicate that the equality "$=$" follows from Lemma 1 / Equation (1) / Property 1 / Theorem 1 respectively (same for inequalities).