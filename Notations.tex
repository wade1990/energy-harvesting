% !TEX root = main.tex
The Transmitter energy arrival instants are marked by $t_i$'s with energy $E^{T}_i$ while the receiver energy arrivals are marked by $r_i$'s with energy $E^{R}_i$ for $i \in \{0,1..\}$. The receiver spends $p_{rcv}$ amount of power to be in '\textit{on}' state and no power when it is in '\textit{off}' state. Hence each energy arrival $E^{R}_i$ can be viewed as it adds $T^{R}_i=\dfrac{E^{R}_i}{p_{rcv}}$ amount of time for which the receiver can be \textit{on}. The maximum amount of time for which the receiver (and hence the Transmitter) can be \textit{on} assuming no energy arriving at the receiver after time '.' is given by function $T^{R}(.)$. It can be easily seen that $T^{R}(t)=\sum_{i=0}^{r_i\le t}T^{R}_i$. Similarly the maximum energy harvested at the transmitter till time '$t$' is given by function $E^{T}(t)=\sum_{i=0}^{t_i\le t}E^{T}_i$. The function $CP(A,B)=\lim_{\epsilon\rightarrow 0}\frac{E^T(A-\epsilon )-E^T(B-\epsilon )}{B-A}$ , $A>B$ denotes the maximum constant power with which transmitter can transmit from time $A$ to $B$.  The rate of bits transmission with power '.', given by function $g(.)$ is assumed to follow the following properties as proposed in \cite{Yang} 
\begin{align}
&P1) g(0)=0\text{ and }\lim_{x\rightarrow \infty} g(x)= \infty.
\\
&P2) g(x)\text{ is concave in nature with } x.
\\
&P3) g(x)\text{ is increasing with } x.
\\ 
&P4) g(x)/(x) \text{ is monotonically decreasing with } x
\\
&\text{ and } \lim_{x\rightarrow \infty} g(x)/x= 0.
\end{align}

For convenience of presentation, we also follow the following convention : we use the notation $\stackrel{L1}{=}$ or $\stackrel{(1)}{=}$ or $\stackrel{P1}{=}$ or $\stackrel{T1}{=}$ to indicate that the equality "$=$" follows from Lemma 1 / Equation (1) / Property 1 / Theorem 1 respectively (same for inequalities).