%!TEX root = OptimalOffline.tex
%input{Siddhartha}
\begin{theorem}

\end{theorem}

































%input{Rushil}
\begin{theorem}
The policy described by the above algorithm is optimal.
\end{theorem}
\begin{proof}
To prove that our policy is optimal, we have to show that it is of the structure described in the previous theorem. \\
That is, $s_{stop} - s_0 \leq T$ and \\
\textbf{Things to write}\\
First we prove that the power allocations in this algorithm are in accordance with \textbf{insert}\\
In the first part of the algorithm, we select the maximum slope at a corner point before $s_{left}$ and after $s'_{start}$ and ending at $s_{left}$. \\
First we try to show that this is also the maximum such slope between any corner point before $s_{left}$ and after $s_{start}$ where $s_{start}$ is the final start point. \\
Suppose it is not. Then we have a corner point between $s_{start}$ and $s'_{start}$ such that we can transmit with a power higher than our maximum between these two points. But, if this were possible, then 
$p_{left}$ itself would have been feasible, which is not the case. (See figure).\\
Now we seek to show that this procedure of selecting maximum slopes going 'backwards' also gives us the minimum slopes going 'forwards', as described in \textbf{insert}.\\
We shall show this by contradiction. Let $s_i$, $s_j$ and $s_k$ be three consecutive corner points where the power of transmission increases, as per our allocation. Now suppose, it is possible to transmit with a lowe power between $s_i$ and some $s'_j$. 
Then the power of transmission between $s_j$ and $s_k$ is not the maximum power since we could transmit at a higher power from $s'_j$ and $s_k$. Which is a contradiction as this is not consistent with out allocation algorithm.\\
Therefore, the allocation policy before point $Q$ is consistent with \textbf{insert}. (See figure)
We can prove similarly for the powers after point $Q$.
\end{proof}




