% !TEX root = OptimalOffline.tex
%input{Siddhartha}
\begin{theorem}
Let a transmission policy to solve Problem 1 is given by power vector $\textbf{p}=[p_1,p_2,..,p_N]$ and the start time of transmission for the corresponding power be given by vector $\textbf{s}=[s_1,s_2,..,s_N]$, for some $N\in \mathbb{N}$. The transmission ends at time $s_{N+1}$. Now such a policy is optimal if and only if it satisfies the following structure.
\label{th_algo1_1}
\begin{align}
&\sum_{i=1}^{i=N}g(p_i)(s_{i+1}-s_i)=B_0\label{claim1}
\\
&s_{N+1}-s_1=T^R_0 \text{ , if } s_1>0; \text{ or } s_{N+1}\le T^R_0 \text{ , if } s_1=0\label{claim2}
\\
&s_{n+1}=\argmin_{t_i: s_n < t_i \le s_{N+1}} CP(t_i,s_n)\text{ and } p_n=CP(s_{n+1},s_n)\label{claim3}
\\
&\exists s_j:s_j\in \textbf{s} \text{ and } s_j=Q\label{claim4}
\end{align}
for $n=\{ 1,2,..,N-1\}$.
\end{theorem}
\begin{proof}
First we show that the optimal policy should have the given structure. The proof follows the method of  contradiction. We establish structure (\ref{claim3}) at first. Assume an optimal policy that satisfies Lemmas 1 to 6 and does not satisfy the given structure (\ref{claim3}). Specifically, say the policy be same as structure (\ref{claim3}) from time $s_{1}$ to $s_n$, for some $n\in \{1,2,..,N\}$ but transmission power right after $s_n$ is not the minimum feasible constant power, i.e.
\begin{align}
p_n>CP(s',s_n)\text{ where } s'=\argmin_{t_i: s_n < t_i \le s_{N+1}} CP(t_i,s_n)\label{claim3_1}
\end{align}

\textit{Case1: }if $s'>s_{n+1}$ for some $n\in \{1,2,..,N-1\}$, then the energy that is used for transmission from time $s'$ to $s_{n+1}$ is given by $E^T(s'-)-E^T(s_{n+1}^-)$ in terms of Lemma 3. We claim that that there must be a time duration from $s'$ to $s_{n+1}$ for which the transmission power is less than $p_n$. If this claim is true then we violate lemma \ref{increasing_power} and hence contradict the assumption. Coming to the claim, if it does not hold i.e. transmission power at all points of time between $s_{n+1}$ to $s'$ is more than $p_n$, then the total energy used during this period can be lower bounded by $p_n(s'-s_{n+1})$. Next, we show that this energy is more that what is harvested during $s_{n+1}$ to $s'$ making it infeasible. As transmitting with $CP(s',s_n)$ power is a feasible between time $s'$ and $s_n$, $CP(s',s_n)(s_{n+1}-s_n)\le E^T(s_{n+1}^-)-E^T(s_{n}^-)$. So, 
\begin{align}
&E^T(s'-)-E^T(s_{n+1}^-) \le E^T(s'-)-E^T(s_{n+1}^-)
\\
&+(E^T(s_{n+1}^-)-E^T(s_{n}^-)-CP(s',s_n)(s_{n+1}-s_n))
\\
&=CP(s',s_n)(s'-s_{n+1})\stackrel{(\ref{claim3_1})}{<}p_n (s'-s_{n+1}).
\end{align}

\textit{Case2: }if $s'<s_{n+1}$, the transmission policy uses $p_n(s'-s_{n})$ energy from time $s_n$ to $s'$. But $E^T(s'-)-E^T(s_{n}^-)=CP(s',s_n)(s'-s_{n})\stackrel{\ref{claim3_1}}{<}p_n(s'-s_{n})$. So, energy used $p_n(s'-s_{n})$ is more than what is harvested making this case infeasible.

%Note that equation (\ref{claim1}) must be followed by the optimal policy as it is a constraint to the optimization problem 1. We move on to prove structure (\ref{claim2}). If $s_1=0$ then $s_{N+1}$ has to be less than or equal to $T^R_0$ due to constraint 1. When $s_1>0$, assume that $s_{N+1}-s_1<T^R_0$. Let $A$ be the first energy arrival such that $E(A)=E^T(A^-)$. Similarly, let $B$ be the last energy arrival at which $E(B)=E^T(B^-)$. Now consider the policy where power vector is given by $\{p_1-\alpha,p_1,p_2,..,p_{N-1},p_N,p_N+\epsilon \}$ and the corresponding time vector be given by $\{s_1-\beta,A,s_2,..,s_{N},B\}$ with the transmission ending at time $s_{N+1}-\gamma$, where $\epsilon$ is a small positive constant, $\beta=\frac{\alpha}{p_1-\alpha}(A-s_1) $, $\gamma= \frac{\epsilon}{p_N+\epsilon}(s_{N+1}-B)$. This policy finishes before the previous policy and hence contradicts its optimality only if we are able to show that it is feasible. Such a policy would be feasible with respect to the energy constraint keeping in mind that transmission with power $p_N+\epsilon $ from time $A$ to $s_{N+1}-\gamma$ was previously never on the boundary of feasibility constraint 1 and similarly for power $p_1-\alpha $. Now we see its feasibility with constraint 2. For every $\epsilon \rightarrow 0^{+}$ we can find a value of $\alpha$ such that the number of bits transmitted in this new policy remains the same as the previous one i.e. $B_0$. So, excluding the common terms from equating the number of bits boils to
%\begin{align}
%&\nonumber g(p_N)(s_{N+1}-B)+g(p_1)(A-s_1)
%\\
%&\nonumber =g(p_1-\alpha)(A-s_1+\beta)+g(p_N+\epsilon)(s_{N+1}-B-\gamma)
%\\
%&\nonumber\implies (s_{N+1}-B)(g(p_N)-g(p_N+\epsilon)\frac{p_N}{p_N+\epsilon})
%\\
%&=(A-s_1)(g(p_1-\alpha)\frac{p_1}{p_1-\alpha}-g(p_1))\label{bits}
%\end{align} 
%Now the total time for which the transmission is \textit{on} is $s_{N+1}-s_1+\beta-\gamma$.


Note that equation (\ref{claim1}) must be followed by the optimal policy as it is a constraint to the optimization problem 1. We move on to prove structure (\ref{claim2}). If $s_1=0$ then $s_{N+1}$ has to be less than or equal to $T^R_0$ due to constraint 1. When $s_1>0$, assume that $s_{N+1}-s_1<T^R_0$. Let $A$ be the first energy arrival such that $E(A)=E^T(A^-)$. Similarly, let $B$ be the last energy arrival at which $E(B)=E^T(B^-)$. Now consider the policy where power vector is given by $\{p_1+dp_1,p_1,p_2,..,p_{N-1},p_N,p_N+dp_N \}$ and the corresponding time vector be given by $\{s_1+ds_1,A,s_2,..,s_{N},B\}$ with the transmission ending at time $s_{N+1}+ds_{N+1}$, where $dp_N>0$ and  $dp_1,ds_1, ds_{N+1}<0$ . This policy finishes before the previous policy and hence contradicts its optimality only if we are able to show that it is feasible. Such a policy would be feasible with respect to the energy constraint keeping in mind that transmission with power $p_N$ from time $A$ to $s_{N+1}$ was previously never on the boundary of feasibility constraint 1 and similarly for power $p_1$. Now we see its feasibility with respect to constraint 2. It can be seen that
\begin{align}
&p_1ds_1=(A-s_1)dp_1\text{ , }p_Nds_{N+1}=-(s_{N+1}-B)dp_N
\end{align}
The number of bits transmitted from time $s_1$ to $A$ is given by $B_1=g(p_1)(A-s_1)$ and similarly, from $B$ to $s_{N+1}$ be given by $B_N=g(p_N)(s_{N+1}-B)$ under the previous policy. Noting that the number of bits sent in the two policies remains same we get,
\begin{align}
&\nonumber dB_1+dB_2=0
\\
&\nonumber\implies g'(p_1)(A-s_1)dp_1-g(p_1)ds_1
\\
&\nonumber +g'(p_N)(s_{N+1}-B)dp_N+g(p_N)ds_{N+1}=0
\\
&\nonumber\implies \frac{-ds_1}{-ds_{N+1}}=\frac{(g'(p_N)p_N-g(p_N))}{(g'(p_1)p_1-g(p_1)}
\end{align}
We can verify that $g'(p)p-g(p)$ is an increasing function of $p$ for $p>0$ due to concavity of $g(p)$. Hence $(-ds_1)\ge (-ds_{N+1})$. The time for which transmission is on in this policy is $s_{N+1}-s_1+ds_{N+1}-ds_1\ge s_{N+1}-s_1$. As $s_{N+1}-s_1<T^R_0$, we can choose arbitrarily small negative value of $ds_{N+1}$ so that $s_{N+1}-s_1\le s_{N+1}-s_1+ds_{N+1}-ds_1<T^R_0$ holds. So the new policy finishes earlier than the previous policy contradicting the optimality. This concludes that $s_{N+1}-s_1=T^R_0$ (if $s_1\neq 0$) in optimal policy.

Next, we prove the sufficiency of the structure. Let the power vector $\textbf{p}$ and time vector $\textbf{s}$ follow the structure. We need to show that this policy is optimal. Assume that there exists another policy given by $\{\textbf{p'},\textbf{s'}\}$ which abides by the Lemma 1-5 and is optimal, but does not follow the structure. We argue next that such a policy is not feasible and hence contradict its optimality. 

\textit{Case1}: If $s_1'>s_1\ge 0$ then by Lemmma  $s_{N'+1}'>s_{N+1}$. So policy $\{\textbf{p'},\textbf{s'}\}$ cannot be optimal. 

\textit{Case2}: Suppose $s_1'=s_1$. Let $s_i'$ be the first epoch for which $p_i'\ne p_i$ for some $i \in \{1,2,..,N\}$. By (\ref{claim3}), $p_i'>p_i$. If $s_{N'+1}'>s_{i+1}$, then the amount of energy used by policy $\{ \textbf{p'},\textbf{s'}\}$ in interval $[s_{i},s_{i+1}]$ is more than policy $\{\textbf{p},\textbf{s}\}$. But by Lemma, $\{\textbf{p},\textbf{s}\}$ uses all energy available by $s_{i+1}$. So policy $\{\textbf{p'},\textbf{s'}\}$ is not feasible with respect to the energy constraint. If $s_{N'+1}'\le s_{i+1}$, then it can be easily verified by property P4 that policy $\{\textbf{p'},\textbf{s'}\}$ transmits strictly less number of bits in interval $[s_i,s_{N'+1}]$ than the other policy in interval $[s_{i},s_{i+1}]$. Both policies being same till $s_i$, we conclude that policy $\{\textbf{p'},\textbf{s'}\}$ transmits less than $B_0$  bits and therefore it is not optimal.

\textit{Case3}: This case argues the infeasibility when $s_1'<s_1$. Unlike other cases this case is more rigorous. The idea of the proof is to show that if we start our transmission early and finish earlier than policy $\{\textbf{p},\textbf{s}\}$, we always take more transmission time which is going to violate the time constraint. First, we establish that the policy $\{\textbf{p'},\textbf{s'}\}$ must be same as policy $\{\textbf{p},\textbf{s}\}$ from epoch $s_2$ to an epoch $s_j$ such that $s_j=\max_{s_i<s_{N'+1}'} s_i$. Let $s_k'=max_{s_i'<s_2}s_i'$ and transmission continues with constant power $p_k'$ till $s_l'$. If $s_l'>s_2$, then transmission with a constant power $\dfrac{E^T(s_l^{,-})}{(s_l'-s_1)} $ from $s_1$ to $s_l'$ is feasible and $\dfrac{E^T(s_l^{,-})}{(s_l'-s_1)}<\dfrac{E^T(s_2^-)}{(s_2-s_1)}=p_1$. This contradicts $\ref{claim3}$. So, $s_l'=s_2$. Now, if $p_l'>p_2$ and $s_j>s_3$, then the amount of energy used by policy $\{\textbf{p'},\textbf{s'}\}$ between $s_2$ and $s_3$ is more than what is harvested. So $p_l'=p_2$ ($s_{l+1}=s_3$) and similarly we can show that $p_{l+1}=p_3$.. ($ s_{l+2}=s_4$..) till epoch $s_j$. By Lemma and (\ref{claim4}) we can be sure that there exists atleast one epoch $s_i$ which belongs to $\textbf{s}$ as well as $\textbf{s'}$ i.e. $j\ge 2$.

Now, consider the following process which creates child feasible policies from policy $\{\textbf{p'},\textbf{s'}\}$. We define two pivots $pv_1$ and $pv_2$. Initially we set $pv_1=s_2'$ and $pv_2=s_{N'}'$. The transmission power right before $pv_1$ is $u$ ($u=p_1'$ initially) and right after $pv_2$ is $v$ ($v=p_{N'}'$ initially). Keeping the policy same from $pv_1$ to $pv_2$ we increase $u$ by a small amount to $u+du$ and decrease $v$ by a small amount to $v-dv$ so that the number of bits transmitted( i.e. $B_0$) remains same under this transformation. Let $s_1'$ change to $s_1'+x$ and $s_{N'+1}'$ change to $s_{N'+1}'+y$ for some $x,y>0$. Following the argument provided while proving the necessary statement of this Theorem, we can conclude that $x>y$ and hence. We denote such a policy by vectors $\{\textbf{p'(x)},\textbf{s'(x)}\}$. Note that $(s_{N'(x)+1}'(x)-s_1'(x))<(s_{N'+1}'-s_1')$. We continue increasing $x$ till either $u=p_2$ (in which case we change $pv_1=s_2$) or $v=p_{N'-1}'$ (where we change $pv_2=s_{N'-1}'$) or $s_{N'(x)+1}'(x)$ hits a epoch, say $t_j$ ($pv_2=t_j$, $v\rightarrow\infty$ in this case). After this, we again start increasing $x$ with changed definitions. We continue this process till $x=s_1-s_1'$  or $u$ becomes equal to $v$. Note that the former stopping criteria will be met at a smaller $x$ than the later one since policy $\{\textbf{p'(x)},\textbf{s'(x)}\}$ shares at least one epoch with policy $\{\textbf{p'},\textbf{s'}\}$ by arguments of previous paragraph. By maintaining these rules we ensure that policy $\{\textbf{p'(x)},\textbf{s'(x)}\}$ abides by Lemma 1-6 and is feasible with energy constraint. Since $s_{N'(x)+1}'(x)-s_1'(x)$ is decreasing with $x$, the policy is also feasible with time constraint. As this is a continuous function on $x$, at $x=s_1-s_1'$ we reach a policy such that $s_1'(x)=s_1$. At $x=s_1-s_1'$, if $s_{N'(x)+1}'(x)\ge s_{N+1}$ then $s_{N'+1}'-s_1'>s_{N'(x)+1}'(x)-s_1'(x)\ge T^R_0$ and policy $\{\textbf{p'},\textbf{s'}\}$ is infeasible with time constraint. If $s_{N'(x)+1}'(x)< s_{N+1}$ then we can follow the arguments in \textit{Case2} to show that policy $\{\textbf{p'(x)},\textbf{s'(x)}\}$ is infeasible, which in turn accounts for the infeasibility of policy $\{\textbf{p'},\textbf{s'}\}$.
\end{proof}

























%input{Rushil}