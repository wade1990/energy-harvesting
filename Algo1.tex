\begin{lemma}
If the reciever has enough energy to stay on for T time, then either the transmitter will transmit for the entire duration T or the transmitter will begin transmission at t=0.
\label{transimission_duration}
\end{lemma}
\begin{proof}
We will prove this by contradiction. Suppose the optimal transmission policy does not begin transmitting at time T and transmits for a duration T'<T.\\
Let $p_1$ be the first power of transmission in this policy. If we reduce this slightly to $p_1-\delta p$, we will have transmitted more bits by time $s_{i_{n-1}}$, where $s_{i_{n-1}}$ is the last energy arrival epoch when the transmission power changes. \\
Therefore at the end we can transmit with a power $p'_n > p_n$ (see figure) and complete our transmission at an earlier time.\\
Thus optimally we can keep lowering our first transmission power until we either exhaust our transmission duration T or we hit the origin.
\end{proof}