% !TEX root = OptimalOffline.tex
\begin{lemma}
The power of transmission in every optimal solution is non-decreasing with time whenever the receiver is \textit{on}.
\label{lemma_increasing_power}
\end{lemma}
\begin{proof}
We prove this by contradiction. The following two cases arise depending on whether the reciever is \textit{on} or \textit{off}.

$Case 1:$ Assume that the power of transmission is $p_1$ from time $A$ to $B$ and then $p_2$ from $B$ to $C$ with $p_1>p_2$ and the receiver is \textit{on} for the entire time $A$ through $C$ as shown in figure \ref{Lemma1}. In this case suppose we transmit at a power $p'=\dfrac{p_1(B-A)+p_2(C-B)}{C-A}$ then the number of bits transmitted would be more over the same time duration due to concavity of $g(p)$ as shown below.
\begin{align}
&g(p_1)\frac{B-A}{C-A}+g(p_2)\frac{C-B}{C-A} \le g(\frac{p_1(B-A)+p_2(C-B)}{C-A})
\\
&\implies g(p')(C-A)\ge g(p_1)(B-A)+g(p_2)(C-B)  
\end{align}
As we can transmit more number of bits during $C-A$ with power $p'$ we can save on the total transmission time since we would have lesser number of bits left to transmit after time $C$. Hence this case cannot be optimal.

$Case 2:$ The receiver is \textit{off} for a certain duration (say from $B$ to $C$) between $A$ and $D$ as shown in figure \ref{Lemma1}. The transmission power is $p_1$ from $A$ to $B$ and $p_2$ from $C$ to $D$. Now, by keeping the reciever off from $A$ to $A+C-B$, if we transmit from $A+C-B$ to $C$ with power $p_1$, instead of from $A$ to $B$ as shown in the figure by the dotted lines, this scenario now boils down to $Case 1$ from time $A+C-B$ to $D$ and hence cannot be optimal.
\end{proof}

\begin{figure}[htb]
\begin{minipage}[b]{0.48\linewidth}
  \centering
  \centerline{\includegraphics[width=4cm]{Lemma1_case1.pdf}}
\end{minipage}
\begin{minipage}[b]{0.48\linewidth}
  \centering
  \centerline{\includegraphics[width=4cm,height=25mm]{Lemma1_case2.pdf}}
\end{minipage}
\caption{Figure showing the two cases of Lemma 1, case 1[left] case2 [right] with $p_1>p_2$}
\label{Lemma1}
\end{figure}

\begin{lemma}
In an optimal solution once transmission has started the receiver is remains \textit{on} until transmission is complete. \label{lemma_nobreaks}
\end{lemma}
\begin{proof}
This is equivalent to saying that there are no breaks during transmission in an optimal solution. Again, we shall prove this by contradiction. Suppose the reciever if \textit{off} for some period after transmission starts. Considering Lemma \ref{lemma_increasing_power} the power of transmission $p_1$ before the break would have to be less than or equal to the power $p_2$ after the break in transmission, as shown in figure . Consider the case where we keep the receiver \textit{off} from time $A$ to $B'=A+C-B$. Now, an energy arrival can occur at the transmitter at any time between $A$ to $D$. If there is no energy arrival then transmitting at a constant rate from $B'$ to $D$ would transmit more bits.

$Case 1:$ If the energy arrival is between $A$ and $B'$, then it can be easily seen that transmitting at a constant rate from $B'$ to $D$ would be better due to concavity of $g(p)$.

$Case 2:$ If the arrival is between $B'$ and $C$ (say $C'$), then again it is easily shown that transmitting at a same rate $p_1$ from $B'$ to $C'$ and  at a constant rate from $C'$ to $D$ would deliver more number of bits.(In the worst case, an energy arrival occurring at $C$ would make this scenario transmit equal number of bits as the original scenario).

$Case 3:$ If there is an energy arrival from $C$ to $D$ (say $D'$), then transmitting at a constant power form $B'$ to $D'$ and then at same rate $p_2$ from $D'$ to $D$ would send more bits to the receiver.

Applying the above scenarios iteratively we could shift the receiver \textit{off} duration $C-B$ to the beginning of transmission and still at worst case transmit equal number of bits in same time duration. Hence having a break in between transmission is always discouraged. This also gives us an idea of why the optimal solution may not be unique.
\end{proof}

\begin{lemma}
In an optimal solution with no breaks, the power of transmission can only change at the time instants when energy arrives at the transmitter. The total energy used for transmission till that instant equals the total energy harvested upto that instant.
\label{lemma_energy_consumed} 
\end{lemma}
\begin{proof}
Keeping in mind Lemma \ref{lemma_increasing_power} and \ref{lemma_nobreaks} the proof of this lemma follows the same structure as that of Lemma 2 in Yang et al. \cite{Yang}. 
\end{proof}	
