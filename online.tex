\textit{Notation:} The starting time of the transmission is denoted by $T_{start}$ and the present time is denoted by $t$. The number of bits and energy left to transmit at any Transmitter energy epoch is represented by $B_{left}$ and $E_{left}$ receptively.
The on-line algorithm that we propose is presented in table \ref{online}.
\begin{table}
\begin{minipage}[b]{8cm}
\caption {On-line Algorithm for energy harvesting transmitter and receiver}
\begin{tabular}{p{7cm}}
\hline \textbf{Input}: Bits to transmit $B_0$; $E^T_i$, $E^R_i$ for $t_i,r_i<t$ where $t$ is the present time instant\\
\hline
\\
\textbf{Initialize:} $T_{start}=\underset{T^R(t)g\Bigg{(} \dfrac{E^T(t)}{T^R(t)}\Bigg{)}\ge B_0}\min t$
\\
\hspace{12mm}$B_{left}=B_0$, $E_{left}=E^T(T_{start})$
\\
\textbf{While} $B_{left} \ge 0$
\\
\hspace{4mm} Transmit at power $\dfrac{E_{left}}{T}$ s.t. $Tg\Big{(}\dfrac{E_{left}}{T}\Big{)}=B_{left}$ 
\\
\hspace{4mm} \textbf{if} $t=t_i$ for any $i$ 
\\
\hspace{7mm} $B_{left}=B_{left}-(t-\max (t_{i-1},T_{start}))g \Big{(}\dfrac{E_{left}}{T}\Big{)}$
\\
\hspace{7mm}$E_{left}=E_{left}-(t-\max (t_{i-1},T_{start}))\dfrac{E_{left}}{T}$
\\
\hspace{4mm}\textbf{end}
\\
\textbf{end}
\\
\hline
\label{online}
\end{tabular}
\end{minipage}
\end{table}
The following lemma can be easily concluded from the definition of the on-line algorithm and hence stated without proof.
\begin{lemma}
The transmit power in the online algorithm is non-decreasing with time after $T_{start}$.
\label{online_power}
\end{lemma}
\begin{theorem}
The competitive ratio of the on-line algorithm presented in Table \ref{online} is 2.
\end{theorem}
\begin{proof}
This is equivalent to saying that the time taken by the on-line algorithm can at max be twice the time taken by optimal off-line algorithm. Let the time taken by the off-line version be $T_{off}$ and the on-line version be $T_{online}$. 

We now show that 
\begin{align}
T_{off}\ge T_{start}
\label{online_time}
\end{align}
This proof follows from contradiction. Let $T_{off}<T_{start}$ and the optimal off-line algorithm transmits with energy in sequence $\{e_1,e_2,..,e_k\}$ for time $\{l_1,l_2..,l_k\} $. Now the number of bits transmitted can be bounded as
\begin{align}
&\sum_{i=1}^{i=k} g\Big{(}\frac{e_i}{l_i}\Big{)}l_i\stackrel{P2}{\le} g\Big{(}\frac{\sum_{i=1}^{i=k}e_i}{\sum_{j=1}^{j=k}l_j}\Big{)} \sum_{j=1}^{j=k} l_j 
\\
&\stackrel{P3,P4}{\le} g\Big{(}\frac{E^T(T_{off})}{T_{off}}\Big{)} T_{off}
\\
&\stackrel{P4}{\le}\lim_{t \rightarrow T_{start}^{-}} g\Big{(}\frac{E^T(t)}{t}\Big{)} t < B_0 \label{online1}
\end{align}
where (\ref{online1}) follows form definition of $T_{start}$. But the off-line algorithm should transmit all $B_0$ bits and hence this concludes that $T_{off}\ge T_{start}$.

Next we estimate the maximum time taken to complete transmission after $T_{start}$ in the on-line algorithm. Let the on-line version transmits at power sequence $\{p_1,p_2,..,p_k\}$ for time $\{l_1,l_2..,l_k\} $. Now,
\begin{align}
&\sum_{i=1}^{i=k}l_ig(p_i)=B_0
\\
&\stackrel{L\ref{online_power}}{\implies}g(p_1)\sum_{i=1}^{i=k}l_i\le B_0
\\
&\implies g\Big{(}\frac{E^T(T_{start})}{T}\Big{)}\sum_{i=1}^{i=k}l_i\le B_0 \text{\hspace{2mm}  s.t. } Tg\Big{(}\frac{E^T(T_{start})}{T}\Big{)}=B_0
\\
&\implies \sum_{i=1}^{i=k}l_i \le T
\end{align}
But, $T\stackrel{P4}{\le} T^R(T_{start})$ as $T_{start}g(\frac{E^T(T_{start})}{T_{start}})\ge B_0=Tg(\frac{E^T(T_{start})}{T})$ and from the definition of $T^R(T_{start})$ it follows that $T^R(T_{start})\le T_{start}$. So we can calculate the competitive ratio as
\begin{align}
&r=\max\dfrac{T_{online}}{T_{off}} = \dfrac{T_{start}+\sum_{i=1}^{i=k}l_i}{T_{off}} \le  \dfrac{T_{start}+T}{T_{off}} 
\\
&\le \dfrac{2 T_{start}}{T_{off}} \stackrel{(\ref{online_time})}{\le} 2
\end{align}      
\end{proof}