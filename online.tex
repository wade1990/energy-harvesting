% !TEX root = main.tex
\textit{Notation:} The starting time of the transmission is denoted by $T_{start}$ and the present time is denoted by $t$. The number of bits and energy remaining to transmit at any Transmitter energy epoch is represented by $B_{rem}$ and $E_{rem}$ receptively.
The on-line algorithm that we propose is presented in table \ref{online}.
\begin{table}
\begin{minipage}[b]{8cm}
\caption {On-line Algorithm for energy harvesting transmitter and receiver}
\begin{tabular}{p{7cm}}
\hline \textbf{Input}: Bits to transmit $B_0$; $\ETx_i$, $E^R_i$ for $t_i,r_i<t$ where $t$ is the present time instant which increments parallely with this algorithm. 
\\
\hline
\\
\textbf{Initialize:} $T_{start}=\min t$ s.t. $\TRx(t)g\Bigg{(} \dfrac{\ETx(t)}{\TRx(t)}\Bigg{)}\ge B_0$
\\
\hspace{12mm}$B_{rem}=B_0$, $E_{rem}=\ETx(T_{start})$, $T=T_{start}$
\\
\textbf{do}
\\
\hspace{4mm} Transmit at power $p$ such that $\dfrac{E_{rem}}{p} g(p)= B_{rem}$
\\
\hspace{4mm} \textbf{if} $t=t_i$ for some $i$ 
\\
\hspace{7mm} $B_{rem}=B_{rem}-(t-\max (t_{i-1},T_{start}))g(p)$
\\
\hspace{7mm} $E_{rem}=E_{rem}+\ETx_i-(t-\max (t_{i-1},T_{start}))p$
\\
\hspace{7mm} $T=t_i$
\\
\hspace{4mm} \textbf{end if}
\\
\textbf{while} $t\le \Big{(}T+\dfrac{E_{rem}}{p}\Big{)}$
\\
\hline
\label{online}
\end{tabular}
\end{minipage}
\end{table}

\begin{lemma}
The transmit power in the online algorithm is non-decreasing with time after $T_{start}$.
\label{online_power}
\end{lemma}
\begin{proof}
From the definition of the algorithm the transmit power only changes when there is a new energy arrival after $T_{start}$. So, if there is no energy arrival the transmit power is same i.e. non decreasing. Suppose there are energy arrivals after $T_{start}$ and for any energy arrival(say $E_{new}$) the power changes from $p_i$ to $p_{i+1}$. Let the energy remaining at start of transmition with power $p_i$ be $E_{rem}$ and bits remaining be $B_{rem}$. The transmition continues for time $l_i$ with power $p_i$. Now, we need to show that $p_i<p_{i+1}$. Form the algorithm we get the following equations. 
\begin{align}
&\frac{g(p_i)}{p_i}=\frac{B_{rem}}{E_{rem}} \label{power_increasing_eq1}
\\
&\frac{g(p_{i+1})}{p_{i+1}}=\frac{B_{rem}-g(p_i) l_i}{E_{rem}+E_{new}-p_i l_i}\label{power_increasing_eq2}
\end{align}
Substituting $g(p_i)$ from (\ref{power_increasing_eq1}) into RHS of (\ref{power_increasing_eq2}) we can see that $\frac{g(p_i)}{p_i}>\frac{g(p_{i+1})}{p_{i+1}}$. Hence by property $P4$ we know that $p_i<p_{i+1}$.
\end{proof}
\begin{theorem}
The competitive ratio of the on-line algorithm presented in Table \ref{online} is 2.
\end{theorem}
\begin{proof}
This is equivalent to saying that the time taken by the on-line algorithm can at max be twice the time taken by optimal off-line algorithm. Let the time taken by the off-line version be $T_{off}$ and the on-line version be $T_{online}$. 

We now show that 
\begin{align}
T_{off} > T_{start}
\label{online_time}
\end{align}
This proof follows from contradiction. Let $T_{off}\le T_{start}$ and the optimal off-line algorithm transmits with energy in sequence $\{e_1,e_2,..,e_k\}$ for time $\{l_1,l_2..,l_k\} $. Now the number of bits transmitted can be bounded as
\\
\textbf{think of a better way to write the proof}
*****
\begin{align}
&\sum_{i=1}^{i=k} g\Big{(}\frac{e_i}{l_i}\Big{)}l_i\stackrel{P2}{\le} g\Big{(}\frac{\sum_{i=1}^{i=k}e_i}{\sum_{j=1}^{j=k}l_j}\Big{)} \sum_{j=1}^{j=k} l_j 
\\
&\stackrel{P3,P4}{\le} g\Big{(}\frac{\ETx(T_{off})}{\TRx(T_{off})}\Big{)} \TRx(T_{off})
\\
&\stackrel{P4}{\le}\lim_{\epsilon \rightarrow 0} g\Big{(}\frac{\ETx(T_{start}-\epsilon)}{\TRx(T_{start}-\epsilon)}\Big{)} \TRx(T_{start}-\epsilon)
\\
&+ g\Big{(}\frac{\ETx(T_{start})-\ETx(T_{start}-\epsilon)}{\TRx(T_{start})-\TRx(T_{start}-\epsilon)}\Big{)} (\TRx(T_{start})-\TRx(T_{start}-\epsilon)) \label{online1}
\end{align}
where (\ref{online1}) follows form definition of $T_{start}$. But the off-line algorithm should transmit all $B_0$ bits and hence this concludes that $T_{off}\ge T_{start}$.
*****
\\
Next we estimate the maximum time taken to complete transmission after $T_{start}$ in the on-line algorithm. Let the on-line version transmits at power sequence $\{p_1,p_2,..,p_k\}$ for time $\{l_1,l_2..,l_k\} $. Now,
\begin{align}
&\sum_{i=1}^{i=k}l_ig(p_i)=B_0\stackrel{L\ref{online_power}}{\implies}\sum_{i=1}^{i=k}l_i\le \frac{B_0}{g(p_1)}\label{bits}
\end{align}
Now, from the definition of $p_1$, $\frac{\ETx(T_{start})}{p_1}g(p_1)=B_0 \le \TRx(T_{start}) g(\frac{\ETx(T_{start})}{\TRx(T_{start})})$. Hence by property $P4$, $\frac{\ETx(T_{start})}{\TRx(T_{start})}\le p_1$. So, the RHS of (\ref{bits}) can be reduced to 
\begin{align}
&\frac{B_0}{g(p_1)} = \frac{\ETx(T_{start})}{p_1} \le \TRx(T_{start})\le T_{start}
\\
\end{align}
where the last inequality followed from the definition of $\TRx(T_{start})$. So we can calculate the competitive ratio as
\begin{align*}
&r=\max\dfrac{T_{online}}{T_{off}} = \dfrac{T_{start}+\sum_{i=1}^{i=k}l_i}{T_{off}} \le \dfrac{2 T_{start}}{T_{off}} \stackrel{(\ref{online_time})}{<} 2
\end{align*}      
\end{proof}
