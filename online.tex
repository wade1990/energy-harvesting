% !TEX root = main.tex
In an online scenario, transmitter and receiver are assumed to have only causal information about energy arrivals i.e. they have no knowledge of future energy harvests. To model a general energy harvesting system, they are further assumed to not have any information about the distribution of future energy arrivals. We propose an algorithm to schedule the transmission of bits in this model. Motivated by \cite{VazeCompetitive}, we use competitive ratio analysis to compare the performance of online algorithm vs. the optimal offline algorithm. In this context, we say that our algorithm is $r$-competitive if for all possible energy arrivals at the transmitter $\ETx(t)$ and all possible `time' arrival $\TRx(t)$ at the receiver, the ratio of time taken by the online algorithm (say $T_{\mbox{\scriptsize{online}}}$) to the optimal offline one (say $T_{\mbox{\scriptsize{off}}}$) is bounded by $r$.
\begin{align}
&\displaystyle\max_{\ETx(t),\TRx(t)\hspace{0.5mm} \forall t}\dfrac{T_{\mbox{\scriptsize{online}}}}{T_{\mbox{\scriptsize{off}}}}\le r
\end{align}
 
\textit{Notation:} The starting time of transmission is denoted by $T_{\mbox{\scriptsize{start}}}$ and the present time is denoted by $t$. The number of bits and energy remaining to transmit at any transmitter energy epoch is represented by $B_{\mbox{\scriptsize{rem}}}$ and $E_{\mbox{\scriptsize{rem}}}$ receptively.

The online algorithm that we propose is presented in Algorithm \ref{algo_online}. The Algorithm waits till time $T_{\mbox{\scriptsize{start}}}$ which marks the first energy arrival at transmitter or `time' addition at receiver such that using the energy $\ETx(T_{\mbox{\scriptsize{start}}})$ and time $\TRx(T_{\mbox{\scriptsize{start}}})$, $B_0$ or more bits can be transmitted.
\begin{equation}
T_{\mbox{\scriptsize{start}}}=\min\ t \ s.t.\  \TRx(t)g\Bigg{(} \dfrac{\ETx(t)}{\TRx(t)}\Bigg{)}\ge B_0.\label{online_T_start}
\end{equation}

To begin with, the transmitter equally divides $\ETx(T_{\mbox{\scriptsize{start}}})$ energy among all $B_0$ bits i.e. the first transmission power $p_1$ is set such that,
\begin{align}
&\frac{\ETx(T_{\mbox{\scriptsize{start}}})}{p_1}g(p_1)=B_0.
\end{align}

By definition of $T_{\mbox{\scriptsize{start}}}$ in \eqref{online_T_start}, we know that transmission with power $p_1$ is going to finish in less than or equal to  $\TRx(T_{\mbox{\scriptsize{start}}})$ time.

If and when energy arrival occurs at the transmitter, the transmission power is changed. The total unused energy left at such an instant, $E_{\mbox{\scriptsize{rem}}}$, is equally divided among the bits left to transmit i.e. $B_{\mbox{\scriptsize{rem}}}$. Note that we do not change our transmission power when there is a `time' arrival at the receiver after $T_{\mbox{\scriptsize{start}}}$, because there is sufficient receiver time available to finish transmission.
\begin{algorithm}
\caption {On-line Algorithm for energy harvesting transmitter and receiver.}
\footnotesize
\label{algo_online}
\begin{algorithmic}[1]
\State \textbf{Input}: Bits to transmit $B_0$; $\ETx_i$, $\TRx_i$ for $t_i,r_i<t$ where $t$ is the present time instant which increments parallely with this algorithm. 

\State $T_{\mbox{\scriptsize{start}}}=\min\ t$ s.t. $\TRx(t)g\Bigg{(} \dfrac{\ETx(t)}{\TRx(t)}\Bigg{)}\ge B_0$
\State $B_{\mbox{\scriptsize{rem}}}=B_0$, $E_{\mbox{\scriptsize{rem}}}=\ETx(T_{\mbox{\scriptsize{start}}})$, $m=T_{\mbox{\scriptsize{start}}}$

\Do
	\State Transmit at power $p$ such that $\dfrac{E_{\mbox{\scriptsize{rem}}}}{p} g(p)= B_{\mbox{\scriptsize{rem}}}$
	\If {$t=t_i$ for some $i$} 
		\State $B_{\mbox{\scriptsize{rem}}}=B_{\mbox{\scriptsize{rem}}}-(t-m)g(p)$
		\State $E_{\mbox{\scriptsize{rem}}}=E_{\mbox{\scriptsize{rem}}}+\ETx_i-(t-m)p$
		\State $m=t_i$
	\EndIf
\DoWhile {$t\le \left( m+\dfrac{E_{\mbox{\scriptsize{rem}}}}{p}\right)$}
\end{algorithmic}
\end{algorithm}

\begin{lemma}
The transmission power in the on-line algorithm is non-decreasing with time after $T_{start}$.
\label{online_power}
\end{lemma}
\begin{proof}
From the definition of the algorithm, the transmission power only changes when there is a new energy arrival at the transmitter, after $T_{\mbox{\scriptsize{start}}}$. 

So, if there is no new energy arrival after $T_{\mbox{\scriptsize{start}}}$, the transmit power is same ($=p_1$) i.e. non decreasing. 

Suppose there are energy arrivals after $T_{\mbox{\scriptsize{start}}}$ and for any energy arrival (say $E_{\mbox{\scriptsize{new}}}$) the power changes from $p_i$ to $p_{i+1}$. Let the energy remaining at start of transmission with power $p_i$ be $E_{\mbox{\scriptsize{rem}}}$ and bits remaining be $B_{\mbox{\scriptsize{rem}}}$. The transmission continues for time $l_i$ with power $p_i$. Now, we need to show that $p_i<p_{i+1}$. From the algorithm we get the following equations.  

\begin{align}
&\frac{g(p_i)}{p_i}=\frac{B_{\mbox{\scriptsize{rem}}}}{E_{\mbox{\scriptsize{rem}}}} \label{power_increasing_eq1}
\\
&\frac{g(p_{i+1})}{p_{i+1}}=\frac{B_{\mbox{\scriptsize{rem}}}-g(p_i) l_i}{E_{\mbox{\scriptsize{rem}}}+E_{\mbox{\scriptsize{new}}}-p_i l_i}\label{power_increasing_eq2}
\end{align}
Substituting $g(p_i)$ from (\ref{power_increasing_eq1}) into RHS of (\ref{power_increasing_eq2}), we can see that $\frac{g(p_i)}{p_i}>\frac{g(p_{i+1})}{p_{i+1}}$. Hence, by monotonicity of $g(p)/p$ from \eqref{property_decreasing}, we know that $p_i<p_{i+1}$.
\end{proof}
\begin{theorem}
The competitive ratio of the online algorithm presented in Algorithm \ref{algo_online} is less than 2.
\end{theorem}
\begin{proof}
This is equivalent to saying that the time taken by the online algorithm can at max be twice the time taken by optimal offline algorithm in the worst case. Let the time taken by the offline policy $\{\bm{p},\bm{s},N\}$  be $T_{\mbox{\scriptsize{off}}}$ and the online version be $E_{\mbox{\scriptsize{online}}}$. Note that $s_{N+1}=T_{\mbox{\scriptsize{off}}}$.

We now show that 
\begin{align}
T_{\mbox{\scriptsize{off}}} > T_{\mbox{\scriptsize{start}}}.
\label{online_time}
\end{align}
We will prove this by contradiction. Suppose $T_{\mbox{\scriptsize{off}}}\le T_{\mbox{\scriptsize{start}}}$. From \eqref{online_T_start}, either $T_{\mbox{\scriptsize{start}}}=t_i$ for some $i$ and/or $T_{\mbox{\scriptsize{start}}}=r_j$ for some $j$.

If $T_{\mbox{\scriptsize{start}}}=t_i$, then as $T_{\mbox{\scriptsize{off}}}\le T_{\mbox{\scriptsize{start}}}$, the maximum energy that can be utilized by the offline algorithm is $\ETx(T_{\mbox{\scriptsize{start}}}^-)=\ETx(T_{\mbox{\scriptsize{start}}})-\ETx_i$. Note that the offline algorithm cannot use energy arrival $\ETx_i$, as using any finite amount of energy for 0 time cannot deliver any bits. 

If $T_{\mbox{\scriptsize{start}}}=r_j$, then the maximum time for which the receiver can be \textit{on} in the offline algorithm is $\TRx(T_{\mbox{\scriptsize{start}}}^-)=\TRx(T_{\mbox{\scriptsize{start}}})-\TRx_j$, as the offline policy has to finish at or before $T_{\mbox{\scriptsize{start}}}$. Note that, in the optimal offline policy, time for which the receiver is \textit{on} is given by $\displaystyle \sum_{i:p_i\neq 0}(s_{i+1}-s_i)$. 
%If $T_{\mbox{\scriptsize{start}}}\neq t_i$ or $T_{\mbox{\scriptsize{start}}}\neq r_j$ then, $\ETx(T_{\mbox{\scriptsize{start}}}^-)=\ETx(T_{\mbox{\scriptsize{start}}})$ or $\TRx(T_{\mbox{\scriptsize{start}}}^-)=\TRx(T_{\mbox{\scriptsize{start}}})$.

Now, the number of bits transmitted by the offline policy $\{\bm{p},\bm{s},N\}$ is given by,
\begin{align}
&\sum_{{\substack{i=1\\p_i\neq 0}}}^{i=N} g(p_i)(s_{i+1}-s_{i}),
\\
&\nonumber \stackrel{(a)}{\le}g\left(\frac{\displaystyle\sum_{i:p_i\neq 0}p_i(s_{i+1}-s_{i})}{\displaystyle\sum_{j:p_j\neq 0}(s_{j+1}-s_{j})}\right)\sum_{j:p_j\neq 0} (s_{j+1}-s_{j}),
\\
&\nonumber\stackrel{(b)}{\le} g\left( \frac{\displaystyle\sum_{i:p_i\neq 0} p_i(s_{i+1}-s_{i})}{\TRx(T_{\mbox{\scriptsize{start}}}^-)} \right)\TRx(T_{\mbox{\scriptsize{start}}}^-), 
\\
&\le g\left(\frac{\ETx(T_{\mbox{\scriptsize{start}}}^-)}{\TRx(T_{\mbox{\scriptsize{start}}}^-)}\right)\TRx(T_{\mbox{\scriptsize{start}}}^-)\stackrel{(c)}{<}B_0.\label{online_eq_2}
\end{align}
%%%&\nonumber\stackrel{(a)}{\le} g\left(\sum_{\substack{i\\p_i\neq 0}}p_i\left(\frac{(s_{i+1}-s_{i})}{\displaystyle\sum_{j:p_j\neq 0}(s_{j+1}-s_{j})}\right)\right) \sum_{\substack{j\\p_j\neq 0}} (s_{j+1}-s_{j}),

where $(a)$ follows from application of Jensen's inequality due to concavity of $g(p)$; $(b)$ follows form the fact that $\displaystyle\sum_{j:p_j\neq 0}(s_{j+1}-s_{j})\le\TRx(s_{N+1})\le \TRx(T_{\mbox{\scriptsize{start}}}^-)$ and $g(p)/p$ is monotonically decreasing; $(c)$ follows form \eqref{online_T_start}. \eqref{online_eq_2} implies that the number of bits transmitted by the offline policy is less than $B_0$. Hence, it amounts to a contradiction and therefore, $T_{\mbox{\scriptsize{start}}}<T_{\mbox{\scriptsize{off}}}$.

%
%If $T_{\mbox{\scriptsize{start}}}=r_j$, then then the maximum time for which the receiver can be \textit{on} is $\TRx(T_{\mbox{\scriptsize{start}}}^-)=\TRx(T_{\mbox{\scriptsize{start}}})-\TRx_j$, as the offline policy has to finish at or before $T_{\mbox{\scriptsize{start}}}$.
%Now, from \eqref{online_eq_4}, the maximum number of bits being transmitted can be bounded by  
%\begin{align}
%&=g\left(\frac{\displaystyle\sum_{i=1}^{i=N}p_i(s_{i+1}-s_{i})}{(s_{N+1}-s_{1})}\right)(s_{N+1}-s_{1}),
%\\
%&=g\left(\frac{\ETx(T_{\mbox{\scriptsize{start}}}}{\TRx(T_{\mbox{\scriptsize{start}}}^-)}\right)\TRx(T_{\mbox{\scriptsize{start}}}^-),
%\\
%&=g\left(\frac{\ETx(T_{\mbox{\scriptsize{start}}}}{\TRx(T_{\mbox{\scriptsize{start}}})-\TRx_j}\right)(\TRx(T_{\mbox{\scriptsize{start}}})-\TRx_j)<B_0
%\label{online_eq_5}
%\end{align}
%where \eqref{online_eq_5} again follows form \eqref{online_T_start}.

Next we estimate the maximum time taken to complete transmission after $T_{\mbox{\scriptsize{start}}}$ in the online algorithm. Let the online version transmit with power sequence $\{p_1,p_2,..,p_k\}$ for times $\{l_1,l_2..,l_k\} $. Now, by Lemma \ref{online_power},
\begin{align}
&\sum_{i=1}^{i=k}l_ig(p_i)=B_0\implies\sum_{i=1}^{i=k}l_i\le \frac{B_0}{g(p_1)}.\label{bits}
\end{align}

From the definition of $p_1$, $\dfrac{\ETx(T_{\mbox{\scriptsize{start}}})}{p_1}g(p_1)=B_0 \le \TRx(T_{\mbox{\scriptsize{start}}}) g\left( \dfrac{\ETx(T_{\mbox{\scriptsize{start}}})}{\TRx(T_{\mbox{\scriptsize{start}}})} \right)$. Hence by \eqref{property_decreasing}, $\dfrac{\ETx(T_{\mbox{\scriptsize{start}}})}{\TRx(T_{\mbox{\scriptsize{start}}})}\le p_1$.\vspace{2pt}
So, the RHS of (\ref{bits}) can be reduced to, 
\begin{align}
&\frac{B_0}{g(p_1)} = \frac{\ETx(T_{\mbox{\scriptsize{start}}})}{p_1} \le \TRx(T_{\mbox{\scriptsize{start}}})\le T_{\mbox{\scriptsize{start}}},
\end{align}
where the last inequality follows from the definition of $\TRx(T_{\mbox{\scriptsize{start}}})$. So we can calculate the competitive ratio for Algorithm \ref{algo_online} as,
\begin{align*}
&r=\displaystyle\max_{\ETx(t),\TRx(t)\hspace{0.5mm} \forall t}\dfrac{T_{\mbox{\scriptsize{online}}}}{T_{\mbox{\scriptsize{off}}}} = \dfrac{T_{\mbox{\scriptsize{start}}}+\displaystyle\sum_{i=1}^{i=k}l_i}{T_{\mbox{\scriptsize{off}}}} \le \dfrac{2 T_{\mbox{\scriptsize{start}}}}{T_{\mbox{\scriptsize{off}}}} < 2.
\end{align*}
%where the last inequality followed from \eqref{online_time}.      
\end{proof}
